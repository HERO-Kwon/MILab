\begin{center}
	\LARGE An In-Air Signature Identification System using Commercial Wi-Fi Devices
\end{center}

\parbox[t]{0.4\textwidth}

{
	\begin{flushright}
		Young-Woong Kwon\\
		School of Electrical and Electronic Engineering\\
		The Graduate School\\
		Yonsei University
	\end{flushright}
}
\vspace{0.1em}

Identity verification using Wi-Fi in-air handwritten signature is a challenging task in signature verification since the shape of the signal varies according to the direction in which the signature is written.
By using the traditional methods, it was difficult to verify identity from signatures entered in various directions. Moreover, limited size of the dataset also limited the training of deep learning models.
In this paper, we propose a method for identity verification from Wi-Fi in-air signatures based on triplet network. Three-channel ConvNet structures is adopted to learn discriminative features from relatively small size of in-air signature datasets. Moreover, we propose a input triplet mining approach based on the kernel and range space learning to faster the convergence speed.
Our experimental results on the Wi-Fi CSI signature dataset shows that the proposed method out-outperforms both the traditional and deep learning based methods by improved verification accuracy and faster loss convergence.

\vspace{\stretch{1}}
\noindent
\hrulefill\\
{\bf Key words : In-air handwritten signature verification, Wi-Fi Channel State Information, triplet network and the Kernel and Range space learning}
\parbox[t]{0.8\textwidth}
{}
