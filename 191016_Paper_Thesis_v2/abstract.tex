\begin{center}
	\LARGE Improving the Triplet Network for Wi-Fi Based Handwritten Signature Verification
\end{center}

\parbox[t]{0.4\textwidth}

{
	\begin{flushright}
		Young-Woong Kwon\\
		School of Electrical and Electronic Engineering\\
		The Graduate School\\
		Yonsei University
	\end{flushright}
}
\vspace{0.1em}

Identity verification using Wi-Fi signals is a challenging task since the shape of the signal varies according both the orientation and the position of the user. In this thesis, a system for identity verification is developed based on the hand gesture signature signals sensed by the Wi-Fi Channel State Information (``CSI'').
A three-channel ConvNet structure is adopted to learn the discriminative features based on a relatively small size in-air handwritten signature dataset. We propose an input triplet mining based on the kernel and range space learning to improve the convergence speed of the triplet network training.
Our experimental results on the Wi-Fi CSI signature dataset shows encouraging accuracy and convergence performances.

\vspace{\stretch{1}}
\noindent
\hrulefill\\
{\bf Key words : In-air handwritten signature verification, Wi-Fi Channel State Information, Triplet network, and the Kernel and Range space learning}
\parbox[t]{0.8\textwidth}
{}
