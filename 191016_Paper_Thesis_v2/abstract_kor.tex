{\large
	\noindent 국문요약}\\[11pt]
\begin{center}
	\vskip 2em \Large 개선된 트리플렛 네트워크를 이용한 와이파이 서명 인식 방법 
\end{center}
\vskip 2em

와이파이 신호를 활용해 입력된 수기 서명으로 사용자의 신원을 확인하는 것은 서명이 입력된 방향에 따라 신호의 형태가 변하기 때문에 해석하기 어려운 과제로 인식되어 왔다.
서명 인식에 전통적인 방법을 사용하면 다양한 방향으로 입력된 서명으로부터 신원을 확인하기 어려웠고, 실험 데이터의 크기가 제한되어 딥러닝 모델의 활용 역시 제한되어 왔다.
본 논문에서는 트리플렛 네트워크에 기반하여 와이파이 신호를 이용해 작성된 서명으로부터 신원을 확인하는 새로운 방법을 제안한다.
트리플렛 네트워크의 채택으로 소규모의 실험 데이터셋에서도 차별적인 특징을 학습하며, 커널과 레인지 공간 학습에 기반한 입력 데이터의 추출로 트리플렛 네트워크의 학습 속도를 빠르게 하였다.
와이파이 서명 데이터셋을 이용한 실험에서 제안된 방법이 서명 검증의 정확도와 손실함수의 수렴 속도에서 전통적인 방법을 능가함을 보여, 다양한 방향으로 입력된 와이파이 신호 기반의 수기 서명을 인식할 수 있음을 검증하였다.


\vspace{\stretch{1}}

\noindent
\hrulefill\\
핵심되는 말:
\parbox[t]{0.8\textwidth}
{수기 서명 확인, Wi-Fi 채널 상태 정보, 트리플렛 네트워크, 커널과 레인지 공간 학습}
