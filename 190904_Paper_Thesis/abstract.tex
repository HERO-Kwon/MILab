\begin{center}
	\LARGE An In-Air Signature Identification System using Commercial Wi-Fi Devices
\end{center}

\parbox[t]{0.4\textwidth}

{
	\begin{flushright}
		Young-Woong Kwon\\
		School of Electrical and Electronic Engineering\\
		The Graduate School\\
		Yonsei University
	\end{flushright}
}
\vspace{0.1em}

Identity verification using Wi-Fi in-air handwritten signature is a challenge task in signature verification since the shape of the signal varies according to the direction in which the signature is written.
By using the handcraft features, it was difficult to verify identity from signatures entered in various directions. Moreover, the limited size of dataset also limited the training of deep learning models.
In this paper, we propose a system for identity verification from Wi-Fi in-air signature signals based on triplet network. Three-channel ConvNet structures is adopted in order to learn discriminative features from the in-air signatures. Moreover, we propose a input triplet mining approach based on the kernel and range space learning to faster the convergence speed.
Our experimental results on in-house Wi-Fi handwritten signature dataset shows the proposed network outperforms handcraft methods and Siamese network by improved verification accuracy and faster loss convergence.

\vspace{\stretch{1}}
\noindent
\hrulefill\\
{\bf Key words : Biometrics, In-air handwritten signature verification, Wi-Fi Channel State Information, triplet network and the Kernel and Range space learning}
\parbox[t]{0.8\textwidth}
{}
