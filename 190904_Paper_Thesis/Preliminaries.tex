\label{chp:Survey}

\section{ConvNets}
% description of convnet
Convolutional Neural Networks is a special case of Multi Layer Perceptron and it has unified feature extractor and classifier in one network. It has been widely applied to visual objects such as image, video or 2D array input. Several factors make ConvNets attractive in image-related tasks.
Local connectivity captures local correlation property of the image. It is applicable by using ConvNet filter. Weight sharing helps to reduce the number of weights in feature maps.
Also, CUDA libraries make the training feature maps easier to reduce training time.

%3.3. CSI
\section{Wi-FI Channel State Information}
CSI captures signal strength and phase information for OFDM subcarriers and between each pair of transmit-receive antennas.
It runs on a commodity 802.11n NIC and records Channel State Information (CSI based on the 802.11 standards.
The CSI contains information about the channel between sender and receiver at the level of individual data subcarriers, for each pair of transmitting and receive antennas.
%[From Halperin End]
% Structure of CSI %[From HC's ELM paper]
In a frequency domain, the CSI of sub-carrier $\mathbf{c}$ between transmitter(Tx) and receiver(Rx) can be modeled as 
$\mathnormal{R}_{c} = \mathbf{H}_{c}\mathnormal{T}_{c} +\mathnormal{N}$ where the $\mathnormal{R}_{c}$ and $\mathnormal{T}_{c}$  denote the received and the transmitted signal vector of dimension $\mathnormal{r}$ and $\mathnormal{t}$, respectively. The $\mathnormal{N}$ is the additive channel noise and $\mathbf{H}_{c}$ is the $\mathnormal{r}\times\mathnormal{t}$ channel matrix. The CSI of sub-carrier $\mathnormal{c}$ can be modeled as follows:
\begin{equation}
    \mathnormal{h}_{c} = \mid\mathnormal{h}_{c}\mid\mathnormal{e}^{\angle\theta},
\end{equation}
where $\mid\mathnormal{h}_{c}\mid$ and $\theta$ represent the amplitude and the phase of the sub-carrier, respectively.
%[From HC's ELM end]
