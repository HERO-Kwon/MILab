\section{Background}
% signature
Signatures written on paper have long been used as an identification tool, and various methods for verifying individuals have been developed~\cite{fahmy2010online,galbally2015line,sanmorino2012survey,sesa2012information}.
% biometric
As mobile devices increase, identity verification using biometrics traits is becoming more widely used.
Compared to the use of complex passwords, using biometric systems to user verification has advantages of increased security, convenience, and accountability~\cite{hutton2004biometrics}.
% camera sensor
With the development of various sensors such as depth camera and mobile camera, in-air signature has also become available~\cite{bailador2011analysis}. However, they had to use a special sensor prepared for signature. ~\cite{jeon2012system,malik20183dairsig} used depth camera, \cite{ketabdar2012magnetic} used mobile device to record position of hand.
% WiFI CSI
On the other hand, some studies use Wi-Fi CSI signals to identify the biometric characteristics of the human body. Since they use commercial devices that are already widespread, they do not need any special input devices.
~\cite{hong2016wfid,liu2015tracking,yousefi2017survey} studied CSI signals to identify the various biometrics characteristics of the body. ~\cite{abdelnasser2015wigest,nandakumar2014wi} used CSI signals to enable users to recognize their gestures.
% CSI signature
Recently, a study was conducted to identify signatures written in the air using CSI signals~\cite{moon2017air}. However, in this study, only the signals entered in one direction were recognized. Due to the nature of the in-air signature, which is difficult to specify the direction in which the signal is input, a verification system is required regardless of the direction in which the signal is entered.
% Deep learning
More recently, there have been studies using deep learning technology to characterize CSI signals. Deep Learning-based models are spotlighted for their automated feature extractors and superior classification capabilities based on them, compared to traditional handcraft models. Deep learning was used to recognize users based on the body shape~\cite{pokkunuru2018neuralwave}, or to identify them with the characteristics shown in their behavior~\cite{shi2017smart}.
% our methods
We used deep learning technology to create a system that can be identity-recognizable even for multi-way air signatures entered with Wi-Fi CSI.
The deep learning model used a triplet network to increase the accuracy of feature extraction while allowing accurate classification, and also improved the model's convergence speed using the kernel and range space running~\cite{toh2018gradient}.

%\newpage
\newpage
\section{Motivation and Contributions}
The main contributions of our work can be summarized as follows:
\begin{itemize}
\item A system to utilize Wi-Fi handwritten signature signals for identity verification is proposed. We used the deep triplet network to obtain the discriminative features in limited training dataset.
\item In training of the triplet network, we propose hard triplet mining strategy using the kernel and range (KAR) space learning to faster the convergence speed.
\item We propose an experimental study using an in-house Wi-Fi handwritten signature dataset collected from 98 subjects.
\end{itemize}

\section{Organization of Thesis}
The rest of this paper is organized as follows: Section 2 introduces the triplet network and the Kernel and Range space learning. Our proposed system is described in Section 3. In Section 4, experimental results is discussed. Conclusions and future works  are presented in Section 5. 
